\chapter{Conclusion and future work}\thispagestyle{EmptyHeader}
\label{chp:5}

In this thesis, a few things were achieved. Firstly, a reasonably comprehensive introduction to linear elasticity theory for topology optimization was presented. Special care was given to all the details to enable a 1-to-1 mapping from the text to the implementation. This is important when developing topology optimization codes and algorithms on unstructured meshes which is what the author here did. This was followed by a presentation of the most common topology optimization algorithms and a detailed presentation of the most common nonlinear programming algorithms used in topology optimization. The author here could not find any single reference that compiles detailed explanations of the theory behind the commonly used nonlinear programming algorithms in topology optimization, together with all their assumptions, strengths and weaknesses and tips on when to use each algorithm. Additionally, a description of all the common paradigms for decision-making under uncertainty was presented.

Realizing the importance and prevalence of CSIMP in topology optimization, all the ways to adapt the penalty step in CSIMP found in literature were reviewed. A gap was found for a general penalty adaptation technique for CSIMP. A flexible and theoretically sound way to adapt penalties was proposed which gave significant speedups in the experiments run. Four common test problems from literature, three 2D and one 3D, were used to test the efficacy of the penalty adaptation with different parameter settings. The main factors affecting the efficacy of the penalty adaptation in the CSIMP algorithm in reducing the number of FEA simulations needed to converge to the final solution were identified. The experimental results demonstrate a significant reduction in the number of FEA simulations required to reach the optimal solution in the decreasing tolerance continuation SIMP algorithm, with exponentially decaying tolerance, with little to no detriment in the objective value and the other metrics used. Finally, a mathematical and experimental treatment of the effect of $x_{min}$ on the convergence of the SIMP algorithm was given with some recommendations for choosing a suitable $x_{min}$. These results were published in \cite{TAREK2020112880}. Some potential future work here is to perform more experiments on more problem classes as well as a larger benchmark set involving more problem classes. To this day, there is a lack of a standard benchmark set for topology optimization across algorithms and programming languages. Preparing such data set will be extremely valuable to the topology optimization society.

Handling load uncertainty significantly increases the computational cost of any algorithm. A comprehensive review of all the literature on handling uncertainty in compliance-based problems was therefore conducted and presented. And a number of exact methods were proposed to handle load uncertainty in compliance-based topology optimization problems where the uncertainty is described in the form of a set of finitely many loading scenarios. This includes mean compliance minimization or a constraint on the mean compliance, minimizing or constraining a weighted sum of the mean and standard deviation of the load compliances as well as minimizing or constraining the maximum load compliance for all the loading scenarios. By detecting and exploiting low rank structures in the loading scenarios, significant performance improvements were achieved using some novel SVD-based methods. The computational complexities of the algorithms proposed were demonstrated and experiments were run to verify the efficacy of the proposed algorithms at reducing the computational cost of these classes of topology optimization problems. The methods presented here are fundamentally data-driven in the sense that no probability distributions or domains are assumed for the loading scenarios. This sets this work apart from most of the literature in the domain of stochastic and robust topology optimization where a distribution or domain is assumed. Additionally, the methods proposed here were shown to be particularly suitable with the augmented Lagrangian algorithm when dealing with maximum compliance constraints. This work was accepted for publication the Structural and Multidisciplinary Optimization journal. Some potential future work here includes developing algorithms for data-driven topology optimization under uncertainty for other classes of topology optimization problems.

Given that the exact methods for handling many loading scenarios require that a low rank exists, more efficient methods were developed when no such low rank exists. In particular, approximation schemes for the mean compliance and a class of scalar-valued functions of the load compliances were developed. The approximation schemes were based on a reformulation of the function approximated as a trace or diagonal estimation problem, opening the door to using many of the available methods for trace or diagonal estimation. The approximation methods were tested on a number of standard 2D and 3D benchmark problems using low and high rank loading scenarios to solve mean compliance minimization as well as minimizing the weighted sum of the mean compliance and its standard deviation. Significant speedups were achieved compared to the naive method as well as the SVD-based method when the rank of the load matrix is high. This work was submitted to the Structural and Multidisciplinary Optimization journal as of the time of writing this thesis. There are a number of possible extensions to this work including trying or developing other trace and diagonal estimators. More generally, developing approximation schemes for other classes of topology optimization problems where there are finitely many loading scenarios is a largely untouched area of research.

Beside the potential future directions suggested above, there are many other potential future works in topology optimization. In particular, one promising direction to purse is the use of differentiable programming and automatic differentiation for topology optimization to simplify and generalize implementations of topology optimization algorithms in multi-physics applications. This is a direction the author is currently pursuing.
