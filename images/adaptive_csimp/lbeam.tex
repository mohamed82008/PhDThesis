\begin{figure}
  \centering
  \resizebox{\textwidth}{!}{
    \begin{tikzpicture}[>=latex]
        \draw[fill,color=gray!70] (-0.2,2) rectangle (3.2,2.5);
        \draw[line width=1.2pt] (0, 2) node(nodeA){} -- (3, 2) node(nodeB){} -- (3, -1) -- (6, -1) node(nodeC){} -- (6, -4) node(nodeD){} -- (0, -4) node(nodeE){} -- cycle;
        
        \draw (nodeA) -- ++(0,1) coordinate (D1) -- +(0,5pt);
        \draw (nodeB) -- ++(0,1) coordinate (D2) -- +(0,5pt);
        \draw [dimen] (D1) -- (D2) node {50 mm};

        \draw (nodeA) -- ++(-1,0) coordinate (D1) -- +(-5pt,0);
        \draw (nodeE) -- ++(-1,0) coordinate (D2) -- +(-5pt,0);
        \draw [dimen] (D1) -- (D2) node {100 mm};

        \draw (nodeC) -- ++(1,0) coordinate (D1) -- +(5pt,0);
        \draw (nodeD) -- ++(1,0) coordinate (D2) -- +(5pt,0);
        \draw [dimen] (D1) -- (D2) node {50 mm};

        \draw (nodeE) -- ++(0,-1) coordinate (D1) -- +(0,-5pt);
        \draw (nodeD) -- ++(0,-1) coordinate (D2) -- +(0,-5pt);
        \draw [dimen] (D1) -- (D2) node {100 mm};

        \draw[->,line width=1pt] (6,-2.5) -- (6.2,-2.5) -- (6.2,-3.15);
        \node (arrowhead) at (6.2,-3.25) {F};
    \end{tikzpicture} \newline
  }
  \caption{L-shaped beam problem}
  \label{fig:LBeam}
\end{figure}
