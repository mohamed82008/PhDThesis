% $Log: abstract.tex,v $
% Revision 1.1  93/05/14  14:56:25  starflt
% Initial revision
% 
% Revision 1.1  90/05/04  10:41:01  lwvanels
% Initial revision
% 
%
%% The text of your abstract and nothing else (other than comments) goes here.
%% It will be single-spaced and the rest of the text that is supposed to go on
%% the abstract page will be generated by the abstractpage environment.  This
%% file should be \input (not \include 'd) from cover.tex.


\chapter*{ABSTRACT}
\normalsize
\setstretch{2}

Topology optimization is a fascinating area of research with numerous unsolved computational challenges. In this thesis, the author aims to advance the research on improving the computational efficiency of common topology algorithms for practical real life problems. Beside the research contributions in this thesis which were published in or submitted to top scientific journals in the field, special care was given to the introduction (chapter 1) of this thesis ensuring it covers much of the theory behind the algorithms and formulations used in topology optimization. A lot of the details presented in the introduction is scattered in multiple resources between computational mechanics books, optimization theory books and papers, and topology optimization literature. This makes it difficult for people starting to learn topology optimization to easily cover the theory needed to do advanced research in the field. An attempt was made to give a reasonably comprehensive coverage of the theory of the finite element method with an emphasis on linear elasticity as well as the theory behind common nonlinear programming algorithms used in topology optimization. Additionally, a presentation of all the common paradigms for decision-making under uncertainty was presented. Topology optimization under uncertainty is a field of research with many unsolved computational problems. This presentation will hopefully help more researchers get started in this field of research more easily.

In chapter 2, the first research contribution of this thesis is presented. In particular, a flexible and theoretically sound way to adapt penalties in the continuation solid isotropic material with penalization (CSIMP) method was proposed which gave significant speedups in the experiments run. Four common test problems from literature, three 2D and one 3D, were used to test the efficacy of the penalty adaptation with different parameter settings. The main factors affecting the efficacy of the penalty adaptation in the CSIMP algorithm in reducing the number of finite element analysis (FEA) simulations needed to converge to the final solution were identified. The experimental results demonstrate a significant reduction in the number of FEA simulations required to reach the optimal solution in the decreasing tolerance CSIMP algorithm, with exponentially decaying tolerance, with little to no detriment in the objective value and the other metrics used. Finally, a mathematical and experimental treatment of the effect of the minimum pseudo-density parameter on the convergence of the CSIMP algorithm was given with some recommendations for choosing a suitable value. These results were published in the prestigious Computer Methods in Applied Mechanics and Engineering journal \citep{TAREK2020112880}.

In chapter 3, the problem of handling load uncertainty efficiently in compliance-based topology optimization problems was tackled. A comprehensive review of all the literature on handling uncertainty in compliance-based problems was presented. And a number of exact methods were proposed to handle load uncertainty in compliance-based topology optimization problems where the uncertainty is described in the form of a set of finitely many loading scenarios. This includes mean compliance minimization or constraining the mean compliance, minimizing or constraining a weighted sum of the mean and standard deviation of the load compliances as well as minimizing or constraining the maximum load compliance for all the loading scenarios. By detecting and exploiting low rank structures in the loading scenarios, significant performance improvements were achieved using some novel methods. The computational complexities of the algorithms proposed were demonstrated and experiments were run to verify the efficacy of the proposed algorithms at reducing the computational cost of these classes of topology optimization problems. The methods presented here are fundamentally data-driven in the sense that no probability distributions or continuous domains are assumed for the loading scenarios. This sets this work apart from most of the literature in the domain of stochastic and robust topology optimization where a distribution or domain is assumed. Additionally, the methods proposed here were shown to be particularly suitable with the augmented Lagrangian algorithm when dealing with maximum compliance constraints. This work was accepted for publication the prestigious Structural and Multidisciplinary Optimization journal.

In chapter 4, approximate methods for handling many loading scenarios with a high rank loading matrix were developed. In particular, approximation schemes for the mean compliance and a class of scalar-valued functions of the load compliances were developed. The approximation schemes were based on a reformulation of the function approximated as a trace or diagonal estimation problem, opening the door to using many of the available methods for trace or diagonal estimation. The approximation methods were tested on a number of standard 2D and 3D benchmark problems using low and high rank loading scenarios to solve mean compliance minimization as well as minimizing the weighted sum of the mean compliance and its standard deviation. Significant speedups were achieved compared to the exact methods when the rank of the load matrix is high. This work was submitted to the Structural and Multidisciplinary Optimization journal as of the time of writing this thesis. 

In chapter 5, a summary of all the findings in this thesis and some potential future work for the author here or for aspiring researchers in topology optimization is presented.

\clearpage
%\newpage
